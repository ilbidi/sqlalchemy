% Created 2017-03-05 Sun 21:34
\documentclass[11pt]{article}
\usepackage[utf8]{inputenc}
\usepackage[T1]{fontenc}
\usepackage{fixltx2e}
\usepackage{graphicx}
\usepackage{longtable}
\usepackage{float}
\usepackage{wrapfig}
\usepackage{rotating}
\usepackage[normalem]{ulem}
\usepackage{amsmath}
\usepackage{textcomp}
\usepackage{marvosym}
\usepackage{wasysym}
\usepackage{amssymb}
\usepackage{hyperref}
\tolerance=1000
\date{\today}
\title{test}
\hypersetup{
  pdfkeywords={},
  pdfsubject={},
  pdfcreator={Emacs 24.4.1 (Org mode 8.2.10)}}
\begin{document}

\maketitle
\tableofcontents

\section{TEST02}
\label{sec-1}
\subsection{Intro}
\label{sec-1-1}
Test su separazione dell'oggetto con la definizione del db e il programma di
utilizzo, copio la struttura del file system dal sito
\url{https://coderwall.com/p/lt2kew/python-creating-your-project-structure}

\subsection{Cosa ho fatto}
\label{sec-1-2}
si creeranno 2 files
\begin{itemize}
\item main.py da lanciare
\item model.py con la struttura delle tabelle
\end{itemize}

la struttura corretta sembra essere creare in file, database.py, con
la configurazione del database da importare neglia altri moduli,
un file, models.py, con le tabelle e un file main.py di lancio.

\section{TEST03}
\label{sec-2}
\subsection{Intro}
\label{sec-2-1}
Ristrutturazione programmi per gestione con unit testing
Ho installato il pacchetto nose con pip install nose
In pratica si tratta di spostare quanto fatto nel main in un pacchetto
di test. Il main potrebbe anche essere lasciato vuoto

\subsubsection{{\bfseries\sffamily DONE} Creare un main}
\label{sec-2-1-1}
Creato un main con un file di configurazione log separato.
PEr richiamare il sistema esegure un import logconfig e caricare il logger con la funzione logging.getLogger, buona pratica è utilizare l'attributo 
\uline{\uline{name}} per definire quale sia il modulo che esegue il log

\subsubsection{{\bfseries\sffamily DONE} Creare nella directory models una struttura tabella}
\label{sec-2-1-2}
Porre attenzione a certi tipi di database, non tutti i campi sono supportati
(o sono supportati in modi diversi, ad esempio in mysql è obbligatorio
inserire la lunghezza di un campo stringa)

\subsubsection{{\bfseries\sffamily DONE} Creare un test che verifichi la creazione e distruzione DB}
\label{sec-2-1-3}
Per alcuni tipi di DB, tipo mysql, in fase di tar down va prima esegito un roll back 
della sessione per essere pena il blocco del test.

per i test ho utilizzato nose, un comando importante (oltre al paramatro -v per vedere che test 
staeseguendo) è --nocapture che permette di vedere a video lo stdout che, normalmente, 
è noascosto da nose.
\section{TEST04}
\label{sec-3}
\subsection{Intro}
\label{sec-3-1}
Strutturazione models per gestione ricezione dati da rilevatori distribuiti
si creeranno nuove classi e nuove tabelle.
A grandi linee la base dati sarà organizzata così

\begin{center}
\begin{tabular}{lll}
Modello & Note & Relazioni\\
\hline
device & Dispositivo che legge i dati (microcontrollore) & \\
devicetype & Tipo di dispositivo & \\
sensortype & Tipo di sensore & \\
data & Dati ricevuti dai dispositivi & \\
\end{tabular}
\end{center}

\subsection{{\bfseries\sffamily TODO} Definire modelli dati e relazioni}
\label{sec-3-2}
\subsubsection{tabella device}
\label{sec-3-2-1}
\begin{center}
\begin{tabular}{lll}
Campo & Tipo & Note\\
\hline
id & Integer & chiave unica autoincrementante\\
code & String(100) & Codice device\\
description & String(256) & Descrizione dispositivo\\
devicetype & devicetype & Tipo dispositivo\\
sensorstype & sensorstype & Lista tipologie sensori dispositivo\\
 &  & \\
\end{tabular}
\end{center}
\subsubsection{tabella devicetype}
\label{sec-3-2-2}
\begin{center}
\begin{tabular}{lll}
Campo & Tipo & Note\\
\hline
id & Integer & chiave unica autoincrementante\\
code & String(100) & Codice devicetype\\
description & String(256) & Descrizione tipo dispositivo\\
 &  & \\
\end{tabular}
\end{center}
\subsubsection{tabella sensortype}
\label{sec-3-2-3}
\begin{center}
\begin{tabular}{lll}
Campo & Tipo & Note\\
\hline
id & Integer & chiave unica autoincrementante\\
code & String(100) & Codice sensortype\\
description & String(256) & Descrizione tipo sensore\\
 &  & \\
\end{tabular}
\end{center}
\subsubsection{tabella data}
\label{sec-3-2-4}

\subsection{{\bfseries\sffamily TODO} Implementare models}
\label{sec-3-3}
\subsection{{\bfseries\sffamily TODO} Implementare test}
\label{sec-3-4}
% Emacs 24.4.1 (Org mode 8.2.10)
\end{document}
